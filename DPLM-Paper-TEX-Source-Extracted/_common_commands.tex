%#######################################################################
% additional useful packages
\usepackage{hyperref}       % hyperlinks
\usepackage{url}            % simple URL typesetting
\usepackage{booktabs}       % professional-quality tables
\usepackage{amsfonts}       % blackboard math symbols
\usepackage{nicefrac}       % compact symbols for 1/2, etc.
\usepackage{microtype}      % microtypography
\usepackage{xspace}
\usepackage{paralist}

% user-defined packages and others
\usepackage{wrapfig}
\usepackage{amsmath}
\usepackage{amssymb}
\usepackage{amsthm}
% \usepackage{graphicx}
% \usepackage{subfig}
\usepackage{url}
\usepackage{paralist}
\usepackage{latexsym}
\usepackage{arydshln}
\usepackage{tabularx}
\usepackage{verbatim}
% \usepackage{algorithm}
% \usepackage{algpseudocode}
\usepackage{CJK}
\usepackage{flushend}
\usepackage{multicol}
\usepackage{multirow,makecell}
\usepackage{color}
\usepackage{colortbl,array}
\usepackage{silence}
\usepackage{arydshln} % for \hdashline
\usepackage{xspace}
\usepackage{soul}
% \usepackage[disable]{todonotes}
\usepackage{pifont}
% \usepackage{caption}
% \usepackage{subcaption}
% \usepackage{todonotes}

\usepackage{arydshln}

% \newcommand{\zhouh}[1]{\textcolor{red}{(zhouh: #1)}}


% \newcommand{\revise}[1]{\textcolor{blue}{#1}}
\newcommand{\revise}[1]{#1}








%#######################################################################
% math commands 
\input{_math_commands}

% \newtheorem{definition}{Definition}
% \newtheorem*{remark}{Remark}


% \usepackage{mathtools}
% \DeclarePairedDelimiter\ceil{\lceil}{\rceil}
% \DeclarePairedDelimiter\floor{\lfloor}{\rfloor}



%#######################################################################
% miscs
\usepackage{tikz}
\newcommand*\circled[1]{\tikz[baseline=(char.base)]{
            \node[shape=circle,draw,inner sep=0.1pt] (char) {#1};}}
\newcommand*\dcircled[1]{\tikz[baseline=(char.base)]{
            \node[shape=circle,draw,inner sep=0pt] (char) {\fontsize{7}{7}\selectfont #1};}}
            
% \newcommand{\circled}[2][6pt]{\raisebox{-#1}{\tikz{\node (F) at (0,0) {#2};\draw[thick](F)\circle(#1);}}}


% Add a period to the end of an abbreviation unless there's one
% already, then \xspace.
\makeatletter
\DeclareRobustCommand\onedot{\futurelet\@let@token\@onedot}
\def\@onedot{\ifx\@let@token.\else.\null\fi\xspace}

\def\viz{\textit{viz}\onedot} \def\Viz{\textit{Viz}\onedot}
\def\eg{\textit{e.g}\onedot} \def\Eg{\textit{E.g}\onedot}
\def\ie{\textit{i.e}\onedot} \def\Ie{\textit{I.e}\onedot}
\def\cf{\textit{c.f}\onedot} \def\Cf{\textit{C.f}\onedot}
\def\etc{\textit{etc}\onedot} \def\vs{\textit{vs}\onedot}
\def\aka{\textit{a.k.a.}\onedot}
\def\wrt{\textit{w.r.t}\onedot} \def\dof{\textit{d.o.f}\onedot}
\def\etal{\emph{et al}\onedot}
\makeatother


% \usepackage{lipsum}

% \newcommand\blfootnote[1]{%
%   \begingroup
%   \renewcommand\thefootnote{}\footnote{#1}%
%   \addtocounter{footnote}{-1}%
%   \endgroup
% }

% url style
% \urlstyle{same}
\urlstyle{tt}


% language direction
\newcommand{\fd}[2]{\textsc{#1}$\rightarrow$\textsc{#2}}
\newcommand{\bd}[2]{\textsc{#1}$\leftarrow$\textsc{#2}}
\newcommand{\bid}[2]{\textsc{#1}$\leftrightarrow$\textsc{#2}}


\newcommand{\hidethis}[1]{}

% colors
\DeclareRobustCommand{\hlnblue}[1]{{\sethlcolor{nblue}\hl{#1}}}
\DeclareRobustCommand{\hlnred}[1]{{\sethlcolor{nred}\hl{#1}}}
\DeclareRobustCommand{\green}[1]{{\sethlcolor{ngreen}\hl{#1}}}
\DeclareRobustCommand{\tgreen}[1]{{\textcolor{ngreen}{\bf #1}}}
\DeclareRobustCommand{\tyellow}[1]{\underline{{\textcolor{cyellow}{\bf #1}}}}
\DeclareRobustCommand{\yellow}[1]{{\sethlcolor{cyellow}\hl{#1}}}
\DeclareRobustCommand{\hlcyan}[1]{{\sethlcolor{cyan}\hl{#1}}}
\DeclareRobustCommand{\hly}[1]{{\sethlcolor{cyellow}\hl{#1}}}

\definecolor{bblue}{HTML}{4F81BD}
\definecolor{oorange}{HTML}{F4C842}
\definecolor{rred}{HTML}{C0504D}
\definecolor{ggreen}{HTML}{9BBB59}
\definecolor{ppurple}{HTML}{9F4C7C}
\definecolor{darkgreen}{HTML}{228B22}
\definecolor{cred}{HTML}{D81B60}
\definecolor{cblue}{HTML}{1E88E5}
\definecolor{cyellow}{HTML}{FFC107}
\definecolor{nred}{HTML}{e41a1c}
\definecolor{nblue}{HTML}{377eb8}
\definecolor{ngreen}{HTML}{4daf4a}
\definecolor{lblue}{HTML}{6C8EBF}


\newcommand{\xmark}{\ding{55}}%
\newcommand{\cmark}{\ding{51}}%
\renewcommand{\checkmark}{\cmark}

% \newcommand{\compactpara}[1]{\vspace{1.5pt}\noindent\textbf{#1}}
% \renewcommand{\paragraph}[1]{\vspace{-1.1mm}\noindent\textbf{#1}}
% \renewcommand{\subsection}[1]{\vspace{0.8mm}\subsection{#1}}

% \setlength{\parskip}{4pt}
% \usepackage{titlesec}
% \titlespacing{\paragraph}{0pt}{-1pt}{\parskip}
% \titlespacing{\section}{0pt}{\parskip}{\parskip}
% \titlespacing{\subsection}{0pt}{\parskip}{\parskip}
% \titlespacing{\subsubsection}{0pt}{\parskip}{\parskip}



\newlength\savewidth\newcommand\shline{\noalign{\global\savewidth\arrayrulewidth
  \global\arrayrulewidth 1pt}\hline\noalign{\global\arrayrulewidth\savewidth}}
\newcommand{\tablestyle}[2]{\setlength{\tabcolsep}{#1}\renewcommand{\arraystretch}{#2}\centering\footnotesize}
\newcommand\blfootnote[1]{\begingroup\renewcommand\thefootnote{}\footnote{#1}\addtocounter{footnote}{-1}\endgroup}

\newcolumntype{x}[1]{>{\centering\arraybackslash}p{#1pt}}
\newcolumntype{y}[1]{>{\raggedright\arraybackslash}p{#1pt}}
\newcolumntype{z}[1]{>{\raggedleft\arraybackslash}p{#1pt}}

\newcommand{\app}{\raise.17ex\hbox{$\scriptstyle\sim$}}
\newcommand{\mypm}[1]{\color{gray}{\tiny{$\pm$#1}}}
\newcommand{\x}{{\times}}
\definecolor{deemph}{gray}{0.6}
\newcommand{\gc}[1]{\textcolor{deemph}{#1}}
\definecolor{baselinecolor}{gray}{.9}
\newcommand{\baseline}[1]{\cellcolor{baselinecolor}{#1}}
\newcommand{\authorskip}{\hspace{2.5mm}}





\newcommand{\citea}[2]{\cite[#1,][]{#2}}
\newcommand{\smallcitep}[1]{\footnotesize\citep{#1}}


\usepackage{color, colortbl}
\definecolor{emerald}{rgb}{0.31, 0.78, 0.47}
% \definecolor{name}{system}{definition}
\definecolor{Gray}{gray}{0.9}
% \definecolor{LightCyan}{rgb}{0.88,1,1}
% \definecolor{Highlight}{rgb}{0.96,0.90,0.92}
\definecolor{Highlight}{rgb}{0.89,0.89,0.94}
\usepackage[first=0,last=9]{lcg}
% \newcommand{\ra}{\rand0.\arabic{rand}}


% \newcommand{\chl}[1]{\cellcolor{Highlight}{#1}}
\newcommand{\chl}{\cellcolor{Highlight}}


\newcommand{\textbi}[1]{\textit{\textbf{#1}}}